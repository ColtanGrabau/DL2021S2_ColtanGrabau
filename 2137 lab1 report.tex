% Digital Logic Report Lab 1
% Created: 2021-20-01, Coltan Grabau

%==========================================================
%=========== Document Setup  ==============================

% Formatting defined by class file
\documentclass[11pt]{article}

% ---- Document formatting ----
\usepackage[margin=1in]{geometry}	% Narrower margins
\usepackage{booktabs}				% Nice formatting of tables
\usepackage{graphicx}				% Ability to include graphics

%\setlength\parindent{0pt}	% Do not indent first line of paragraphs 
\usepackage[parfill]{parskip}		% Line space b/w paragraphs
%	parfill option prevents last line of pgrph from being fully justified

% Parskip package adds too much space around titles, fix with this
\RequirePackage{titlesec}
\titlespacing\section{0pt}{8pt plus 4pt minus 2pt}{3pt plus 2pt minus 2pt}
\titlespacing\subsection{0pt}{4pt plus 4pt minus 2pt}{-2pt plus 2pt minus 2pt}
\titlespacing\subsubsection{0pt}{2pt plus 4pt minus 2pt}{-6pt plus 2pt minus 2pt}

% ---- Hyperlinks ----
\usepackage[colorlinks=true,urlcolor=blue]{hyperref}	% For URL's. Automatically links internal references.

% ---- Code listings ----
\usepackage{listings} 					% Nice code layout and inclusion
\usepackage[usenames,dvipsnames]{xcolor}	% Colors (needs to be defined before using colors)

% Define custom colors for listings
\definecolor{listinggray}{gray}{0.98}		% Listings background color
\definecolor{rulegray}{gray}{0.7}			% Listings rule/frame color

% Style for Verilog
\lstdefinestyle{Verilog}{
	language=Verilog,					% Verilog
	backgroundcolor=\color{listinggray},	% light gray background
	rulecolor=\color{blue}, 			% blue frame lines
	frame=tb,							% lines above & below
	linewidth=\columnwidth, 			% set line width
	basicstyle=\small\ttfamily,	% basic font style that is used for the code	
	breaklines=true, 					% allow breaking across columns/pages
	tabsize=3,							% set tab size
	commentstyle=\color{gray},	% comments in italic 
	stringstyle=\upshape,				% strings are printed in normal font
	showspaces=false,					% don't underscore spaces
}

% How to use: \Verilog[listing_options]{file}
\newcommand{\Verilog}[2][]{%
	\lstinputlisting[style=Verilog,#1]{#2}
}




%======================================================
%=========== Body  ====================================
\begin{document}
	
	\title{ELC 2137 Lab \#\#: Git and LaTeX Intro Lab 1}
	\author{Coltan Grabau}
	
	\maketitle
	
	
	\section*{Summary}
	
	This lab, the main objective was to learn about the two main tools that are used for programming: Git and LaTex. The basics of the Git process is to allow a user to upload their file and later pull this file and work on it, or to allow another programmer to pull the file, edit it, and then push it back into the main file. These files can either be local or remote meaning either the files are on the computers hard drive or acceciable from anywhere. The purpose of LaTex is to allow a user to edit a report and customize every option. These reports, as shown below, can have images, code, lists, tables, and section headings.
	
	
	\section*{Q\&A}
	
	\begin{enumerate}
		\item What is your GitHub user name?
		
		ColtanGrabau
		
		\item What  LaTeX  environment  produces  a  bulleted(non-numbered) list?
		
		Itemize 
		
		\item Write  the  equation y(t) = 1/2 $e^t$ using  La-TeX equation formatting.
		
		$y(t) = \frac{1}{2}e^t$
		
		\item What is the shortcut key for compiling your La-TeX document?
		
		F6
		
	\end{enumerate}
	
	
	\section*{Results}
	\begin{figure}[ht]\centering
		\begin{center}
			\begin{tabular}{c|c|c}
				\toprule
				Binary & Hex & Decimal \\
				\midrule
				0000 & 0 & 0 \\
				0010 & 2 & 2 \\
				0100 & 4 & 4 \\
				0110 & 6 & 6 \\
				1000 & 8 & 8 \\
				1010 & A & 10 \\
				\bottomrule
			\end{tabular}
		\end{center}
		\includegraphics[width=1\textwidth,trim=18.5cm 15.5cm .5cm 4.5cm,clip]{lab1_example_screenshot}
		\caption{Table and simulation waveform to reproduce}
		\label{fig:Simulation}
	\end{figure}

	
	
	\section*{Code}
	
	\Verilog[caption=File-included Verilog code example,label=code:file_ex]{lab1_example_code.sv}
	
	
\end{document}
